\chapter{Oscar95}

Oscar95 je školní výukové robotické rameno od firmy Elcom Education s.r.o. řízené čtyřmi krokovými motory. Rameno bylo zkonstruováno v roce 1995. Původně sloužilo jako pomůcka pro výuku automatizace a robotiky. Dalo se řídit buď ručně pomocí speciálního ovladače s tlačítky a nebo za pomocí programu v počítači. Program v počítači umožňoval studentům si vyzkoušet jednoduché řízení robota pomocí programu. Program mohli studenti psát v jazyce PASCAL, C++, BASIC, nebo také Baltazar. Poslední laboratorní práce s ním byla provedena v roce 2009 -- viz. \ref{fig:oscar_old}  na straně 
\pageref{fig:oscar_old}. Od té doby čeká na modernizaci. V některých aspektech je vzhledem ke stáří poměrně neaktuální.
Například deska s řídící elektronikou využívající ke komunikaci ještě starý canon 25 konektor. Krokové motory jsou řízené pomocí LM298 driveru, který je řízen ještě TTL logickými hodnotami. Napájení desky bylo realizováno pomocí externího 12\,V zdroje stejnosměrného napětí. Rameno umožňuje pohyb 360° kolem základny, svírání předních čelistí a pohyb svých dvou kloubů. Pohyb v kloubech a svírání přední čelisti je realizován pomocí závitových tyčí. Rameno je vyrobené převážně z hliníku, až na pár ocelových dílů a mosazné závitové tyče, aby byla celá konstrukce, pokud možno co nejlehčí. Kontrukce na první pohled působí robustně a odolně. Rameno dokáže uzvednout předmět o hmotnosti až 100 g. Rameno je vybaveno mechanickými dorazy, ale zároveň disponuje i optickými senzory koncových dojezdů. Původní deska byla vzhledem ke stáří a nekompatibilitě s dnešní pokročilou technikou nahrazena modernější. \cite{Oscar95} \cite{staveb-robot} \cite{Laborka-2009}

		

