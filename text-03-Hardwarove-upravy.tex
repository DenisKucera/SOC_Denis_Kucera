\chapter{Hardwarové úpravy}

\section{Úpravy desky}Už od začátku práce na projektu Oscar95 bylo jasné, že původní desku plošného spoje z roku 1995 bude nutné nahradit. O to se postaral můj před\-chůd\-ce Ondřej Kratochvíl ve své maturitní práci. Nyní je k dispozici univerzální deska, se kterou můžu pracovat, popřípadě ji dále vyvíjet. \cite{Universalstepperboard} V následující kapitole se budu věnovat svým úpravám desky, které bylo nutné dodělat. Jednou z nejdůležitějších elektrotechnických úprav je bezpochyby nové napájení desky. Původně musela být nová deska napájena 12\,V zdrojem a zároveň USB kabelem 5\,V z počítače. 12\,V bylo hlavní silové napájení pro drivery krokových motorů a 5\,V pro mikrokontrolér. Přidal jsem proto stabilizátor napětí, který snižuje napětí z 12\,V na 5\,V. Připojil jsem mezi stabilizátor a hlavní napájení ze zdroje diodu. Dioda je tam proto, aby v režimu napájení z USB nedocházelo únikům do desky. Teď může deska fungovat i bez připojeného USB kabelu. Důležitou maličkostí se stalo přidání kondenzátoru k bootovacímu vývodu mikrokontroléru. Nebylo už nadále zapotřebí mačkat při každém nahrávání nového kódu mačkat bootovací tlačítko. Dalším důležitým prvkem byla detekce připojení hlavního napájení 12\,V. To jsem realizoval pomocí optočlenu připojeném na řídící vývod mikrokontroléru. Mikrokontrolér tak dostavá informaci pokaždé když dojde k připojení, nebo odpojení hlavního napájení. Poslední úpravy proběhly v rámci přidávaní nových signalizačních prvků.
\enlargethispage{20mm}
Například připojení piezo měniče, LED diod a  vypínacího/zapínacího tlačítka.  

\section{Konstrukční úpravy}Bylo potřeba dát robota do pořádku i po stránce hardwarové. Seřídit celé rameno, zkontrolovat všechny mechanické prvky, popřípadě opravit menší nedostatky z výroby. Důležitým úkolem bylo vymyslet upevnění řídící desky do základny ramene. Padlo několik návrhů, jak desku umístit. Nakonec jsme zvolili umístění mezi sloupky podstavy. Návrhů samotného řešení jsem udělal několik. Mým cílem bylo desku připevnit bez mechanického zásahu do podstavy. Počítal jsem s tím, že do budoucna pro případnou novou desku bude podstava zase jiná. Zvolil jsem proto tvar, který kopíruje vnitřní obvod podstavy. Na tuto plochu jsem potom přidal distanční sloupky pro dostatečné odsazení desky. Výsledná plocha držáku byla veliká. Se spolužákem jsme proto poté ve Sliceru upravili desku tak, aby se na její výrobu spotřebovalo co nejméně materiálu, ale zároveň byla pevná. Výsledek jsem poté upevnil u krajů základny tavnou pistolí. Poslední velmi důležitou úpravou základny Oscara jsem se inspiroval u spolužáka Šimona Skládaného. Vymyslel přední panel pro konektory vyvedené z desky. Pro umístění panelu jsem zvolil díru, která v základně zůstala po demontáži původní desky s paralelním portem. Tato díra má rozměry přibližně 8,3x6,3\,mm, což je pro uchycení dostačující. Toto řešení se mi líbilo, a proto jsem jeho nápad použil a přidal své vlastní úpravy. Panel jsem navrhoval v programu Solidworks, ve kterém se mi pracuje dobře. Vyvést ven jsem chtěl hlavně konektor USB z mikrokontroléru ESP32. Pokaždé, když by bylo potřeba nahrát nový program, muselo by se odšroubovat celé víko základny, a to je velmi nepraktické. Pro svou mechanickou odolnost a velmi dobré rozměry jsem si nakonec vybral konektor USB typ B ve standardní velikosti. Následně jsem vyrobil redukci a vymodeloval pro ni místo na panelu. Dalším důležitým prvkem je vyvedené napájení desky. To jsem vyřešil pomocí přidání napájecího DC power Jack konektoru. Pro jeho upevnění stačilo do panelu udělat díru o správném průměru. Poslední periferie, které jsem chtěl přidat na přední panel, byly signalizační prvky, tlačítko ON/OFF a dvě LED diody.\\ Stejně jako u napájecího konektoru stačilo pouze zhotovit díry se správným průměrem. Celý model panelu jsem nechal vytisknout u kamaráda na 3D tiskárně.

\section{Další úpravy robota}Na závěr jsem se ještě rozhodl dodělat chybějící podložky, vymodelovat nové nástavce do čelistí a připojit DIAG pin do driveru TMC2209. Komponenty jsem opět modeloval v programu Solidworks. U podložek jsem se snažil, aby byly co nejvíce podobné těm původním. Nejenom z hlediska vzhledu, ale i materiálu. Materiál pro tisk jsem zvolil ABS. Následně jsem nechal díl vytisknout na školní 3D tiskárně. Stejný postup jsem zopakoval i pro nástavce do čelistí. S tím rozdílem že jsem použil jiný materiál. Nástavce jsem nechal vytisknout z pružného flex materiálu, který se následně dokáže tvarem přizpůsobit. To se bude hodit v případě, že předmět nebude mít dokonale hladký povrch. Poslední významnou úpravou bylo připojení DIAG pinu do mikrokontroléru ESP32. Bez něj by totiž nefungovala funkce Stallguard\,\footnote{popsáno v kapitole 5}. Z driveru jsem drátem vyvedl tyto piny a připojil je na GPIO piny 19, 21, 22, 23, mikrokontroléru ESP32. V programu jsem následně piny nastavil a vytvořil pro ně hardwarový interrupt. Tento interrupt čeká na pulzy přicházející z DIAG pinu. \cite{BIGTREETECH-TMC2209}


	\begin{figure}
		\begin{center}
			\includegraphics[scale=0.25]{img/nastavec.png}
			\caption{Vymodelovaný nástavec do čelistí robota (vlastní fotografie)}
			\label{fig:nastavec}
		\end{center}
		\vspace{-7mm}
	\end{figure}
