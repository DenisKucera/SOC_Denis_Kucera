\chapter{Ovládání pomocí aplikace}
Součástí každého moderního zařízení je umožnění řízení z mobilního telefonu. Jak již bylo zmíněno Oscar95 byl původně řízen speciálním ovladačem. Podobný ovladač jsem měl v plánu přidat také. Bylo by ale zbytečné mít v dnešní době podobný ovladač. Zjistil jsem, že existuje aplikace RBcontroller a prostředí RBGridUI vyvinuté Robotárnou, pobočkou DDM Helceletova. Rozhodl jsem proto implementovat do projektu tyto moderní prvky. Vize byla taková, že by student mohl kompletně řídit Oscara95 z telefonu nebo PC. V reálném čase by student mohl řídit pohyb robota, kontrolovat jeho pozici a případně ukládat dané souřadnice do mezipaměti a vytvořit tím souvislou trasu pohybu pro robotické rameno. Následně by student tuto trasu mohl upravovat a přidávat do ní nové body. \cite{RBController}
\section{Představení RBGridUI}
RBCgridUI je rozhraní vyvinuté také Robotárnou, které zprostředkovává 
komunikaci mezi mobilní aplikací a mikrokontrolérem ESP32. Komunikace probíhá 
pomocí wifi, nebo bluetooth připojení. Celé prostředí funguje za použití wifi síťového protokolu a webserveru běžícího na ESP32, na který 
se uživatel s aplikací připojuje. Tento webserver pracuje na samostatném jádře procesoru. Na 
druhém jádru se provádí hlavní program. Hlavní program potom načítá vzhled a funkce 
definované uživatelem v knihovně \textit{layout.hpp}. Velkou výhodou je možnost navrhnout si rozhraní pro ovládání robota uži\-va\-tel\-sky. K tomu slouží internetová stránka, ve které si lze svůj vlastní layout snadno vytvořit 
pomocí přidávání různých funkčních prvků. Celé schéma si následně můžeme upravit sami 
pomocí přidání vlastních stylů. Následně algoritmus vygeneruje kód, který pak stačí už jen vložit 
do správné složky. Práce s tímto prostředím je velmi jednoduchá a efektivní. \cite{RBGridUI}


\begin{figure}
		\begin{center}
			\includegraphics[scale=0.5]{img/rbgridui.jpg}
			\caption{Vzhled vytvořeného prostředí aplikace (vlastní fotografie)}
			\label{fig:rbgridui}
		\end{center}
		\vspace{0mm}
\end{figure}

\section{Aplikace RBController}
Samotná mobilní aplikace RBController je rovněž dílem Robotárny. Aplikace funguje na telefonech s operačním systémem Android a na počítačích s Windows nebo Linux. Umožňuje posílat instrukce z mobilního telefonu prostřednictvím bluetooth, nebo wifi. Prostředí aplikace je jednoduché a efektivní. Připojíte se na zařízení, vyberete si uživatele s jeho zařízením a následně aplikace spáruje uživatele se zařízením. Jedinou nevýhodou je, že aplikaci si nemůžou nainstalovat uživatelé iOS operačních systémů.  \cite{RBController}

